%!TEX root = disclosure_game.tex
\begingroup
\let\clearpage\relax
\let\cleardoublepage\relax

\pdfbookmark[1]{Abstract}{Abstract}


\section*{Abstract}

Background:
Roots in agents, decision theory, game theory.
We draw together methodologies from game theory, agent based modelling, and decision theory to explore the process of decision making around disclosure. This is framed in the context of pregnant women disclosing their drinking behaviour to their midwives.

Objective: 
To pimp out this method, and maybe get some insight into reality.
Address the perceived ad hoc character with strongly theoretically underpinned agents.

Methods: 
Made a game, made a simulation, did a sensitivity analysis.

Results: 
Look! Chaos! Look! Qualitative trends! Look! More data needed.

Conclusions:
This.. might actually kind of work??

Comments: 
Lesson learned: data should drive modelling.

This dissertation presents a method for modelling disclosure behaviour
by treating the interaction as paired signalling games played by decision
theoretic agents. Two theories of decision making - Bayesian risk
minimisation, and \ac{CPT} - are investigated, and a simulation developed
using Python.

The feasibility of the method is examined through a case study, which
considers the disclosure of the drinking behaviour of pregnant women
to their midwives. 

The essence of the scenario is that there is considered to be a long
term benefit to disclosure - common in the healthcare arena, but an
opportunity cost associated with disclosure. In the case study, this
is conceptualised as arising from the perceived undesirability of
drinking while pregnant. More generally this could derive from any
imbalance in the long term benefit and the opportunity cost, for example
the subjective benefit of a cigarette in the near future, versus the
discounted benefit of better health later.

Theories of decision are driven by the weighing of probabilities and
subjective gains or losses. In this case, the probabilities are generated
by individual agents based on their initial preconceptions and their
experiences across the simulation, using Bayesian inference.

The Bayesian risk minimising model is able to reproduce qualitative
trends around increased honesty over appointments, and a negative
impact of harsh judgement of drinkers on disclosure. The \ac{CPT}
model is less successful, which may be a result of improper, or excessively
homogenous parameters, in combination with unrealistic payoffs.

A global sensitivity analysis is also conducted using Gaussian Emulation
Machines, and finally recommendations for further work are dervived, along
with a few key recommendations for practice - assume people are being honest,
and be non-judgemental.

\vfill{}


\endgroup

