%!TEX root = disclosure_game.tex
\section{Alcohol, and Disclosure in the Maternity Setting}
\label{sec:alcohol_disclosure}

This section presents a brief overview of literature focusing on the impact
of drinking behaviour in pregnancy, and factors affecting disclosure
behaviour in the midwifery context.


\subsection{Impact of Alcohol\label{sub:Impact-of-Alcohol}}

Distinct from stigma attached to alcohol consumption in pregnancy,
is the question of the real impact on woman and baby both in the \gls{antenatal}
period, and beyond. While the canonical example of alcohol linked
disorders is \ac{FAS}, and others on that spectrum, heavy drinking
during pregnancy has been mooted as a factor in a variety of negative
health outcomes. 

The impact of moderate alcohol consumption in pregnancy is more contested.
For example, \citet{Andersen2012} examined moderate drinking in a
large Danish cohort study, finding a significant increase in the risk
of spontaneous abortion at low levels of consumption early in pregnancy.
\citet{Savitz2012} questioned the extent to which this can be interpreted
as a causal connection, noting that there is a known relationship
between absence of morning sickness, and spontaneous abortion, and
suggesting that this may explain much of the difference in risk. \citet{Kesmodel2002}
examined the relationship between alcohol consumption and still-birth,
finding that increased consumption lead to an increase in risk to
the baby, but in contrast to \citet{Andersen2012} this was significant
at \gls{term}. 

Considering longer term negative outcomes, a meta-analysis by \citet{Latino-Martel2010}
examined the potential for maternal alcohol consumption in pregnancy
to feature as a risk factor for onset of childhood leukaemia, finding
that any alcohol consumption was associated with an increased risk
of childhood \ac{AML}, but note the rarity of the condition as a
limitation.

\citet{Huizink2006} reviewed literature looking at the impact of
moderate consumption on neurodevelopmental and cognitive outcomes,
concluding that maternal consumption can be a contributing factor
to \ac{ADHD}, and impairments to learning and memory. They subsequently
suggest that the underlying mechanism is not specific to alcohol consumption,
but a more general phenomena arising from perturbations to foetal
conditions \citep{Huizink2009}, but caution that methodological issues
in many of the studies reviewed may undermine this hypothesis.

Contrary to this, a meta-study by \citet{Gray2006} found there was
insufficient evidence to suggest any harm arising from moderate (under
1.5 UK units per day) alcohol consumption. This ties to the current
guidance from the \ac{NICE} \citep{NICE2010a}, advising that women
should avoid drinking at all in at least the first three months of
pregnancy, and no more than 1-2 units once or twice a week if they
do. In giving this advice, \ac{NICE} acknowledge that the risks to
the foetus from alcohol are a somewhat contentious subject, concluding
that the evidence of harm is inconclusive, but that this is not sufficient
to rule out the risk of negative outcomes. This tension is reflected
by earlier guidance from the \ac{RCOG} \citep{Gynecologists1996}
suggesting no evidence of harm below 15 units per week, and subsequent
criticism by \citet{Guerri1999}, who suggest that this might be interpreted
as legitimising \gls{binge}, while noting several studies indicating
adverse affects linked to even a single drink per day (e.g. \citet{Day1990}).
A subsequent \ac{RCOG} statement \citep{RoyalCollegeofObstetriciansandGynaecologists2006}
revised the recommendations to incorporate newer findings, advising
that there is no known safe threshold for drinking in pregnancy, and
highlighting binge drinking as of particular concern.

There has recently been an increased interest in the impact of binge
drinking, as a distinct pattern of consumption, with a wide variety
of negative outcomes reported by \citet{Maier2001}, although a significant
portion of their evidence base is drawn from animal studies which
augers for caution in generalising findings to humans. \citet{Strandberg-larsen2008}
explored links between binge drinking, and stillbirth, reporting a
statistically significant increase in risk associated with more than
three antenatal binge episodes. \citet{Sun2009} looked at seizure
disorders in children whose mothers binged during pregnancy. They
reported significantly greater risk of both neonatal seizures (\textasciitilde{}3
fold) and epilepsy (1.81 fold) associated with binge drinking between
11 and 16 weeks, but emphasised the exploratory nature of the results,
and need for replication. In terms of neurodevelopmental outcomes,
\citet{Streissguth1994} found a dose dependent association with scores
on timed word, and arithmetic tests in fourteen year olds with a stronger
association where bingeing occurred. A review by \citet{Henderson2007}
cautiously supports the contention that binge drinking has a neurodevelopmental
impact, but found no consistent support for adverse outcomes in pregnancy
(e.g. stillbirth, miscarriage, etc.) and note a paucity of studies
in the area. \citet{Meyer-Leu2011} considered the neonatal period,
finding that both moderate and binge drinking were associated with
an increased trend towards neonatal asphyxia. They also noted a large
number of contradictory findings and raising methodological concerns
about the studies reviewed by \citeauthor{Henderson2007}. \citet{Barr2006}
contend that binge drinking may also contribute to psychiatric issues
in the later life of offspring, although in this case their findings
are confined to individuals with \ac{FAS}, which may in itself be
a confounding factor, rather than indicating a directly causative
relationship between antenatal binge drinking and subsequent psychiatric
disorder in offspring.

Overall, there is a distinct lack of consensus on what, and how extensive,
the effects of drinking on the immediate and long term health outcomes
are for the child. 


\subsection{Disclosure}



The issue of disclosure is central to the model presented here, in
particular self-report by women of information that might disadvantage
them, or be expected to do so in the immediate term. In general, the
consensus is that alcohol self-reports have acceptable validity in
the research context \citep{DelBoca2000}, but do not correspond perfectly
to alternative methods. \citet{DelBoca2003} claim that the validity
is generally accepted, and suggest that the current focus lies on
what factors and processes underlie the discrepancies rather than
questioning determining their existence. In this instance, the conjecture
is that the information is in some way stigmatising; that, following
\citet{Goffman1999}, disclosure equates to revelation of the mark.
This is not immediately contentious, for example \citet{Gomberg1988}
identified stigma surrounding alcohol abusing women in particular,
an issue also highlighted by \ac{IAPT} guidance \citep{IAPT2012},
as well as a number of other studies relating response effects to
perceived negative consequences \citep{Langenbucher2001,DelBoca2000,Blair1977}.
In the maternity context, \citet{Radcliffe2011} identifies stigma
pertaining to substance misusing women amongst staff, and suggests
that this may represent a barrier to appropriate treatment; similarly,
\ac{NICE} guidance on pregnancy and complex social factors \citep{NICE2010}
recognises concern about the attitude of staff as a source of anxiety
in pregnant women who misuse substances.

Stigma, or fear of a judgemental response on the part of the practitioner
should not however be taken uncritically to explain inaccurate reporting
by patients. While recent \ac{NICE} public health guidance advocates
routine alcohol misuse screening as a part of all practice \citep{NationalInstituteforHealthandCareExcellence2010},
there is no specific policy for routine antenatal care beyond providing
information on possible impacts of alcohol consumption \citep{NICE2010a}.
\ac{NICE} guidance on pregnancy and complex social factors \citep{NICE2010}
does specifically address women who misuse alcohol, but presupposes
knowledge of the problem through medical history, or via other services.
Taken in concert with the potential for harm from even moderate alcohol
use (section \ref{sub:Impact-of-Alcohol}), this suggests that much
of the onus is on the patient to volunteer information.

Where screening is used, \citet{Kaskutas2000} note that the most
basic method, i.e. number of standard drinks consumed, can lead to
inaccurate estimates of consumption arising from inability to relate
the concept of a standard drink, to actual consumption. This is compounded
by the impact of memory effects on recall over a number of days \citep{Stockwell2004},
and a lack of consistency in the standard drink measure \citep{Turner1990}.
Alternative screening tools, for example \acs{AUDIT}, and \acs{T-ACE}
are available and have been shown to perform well in identifying problematic
levels of drinking \citep{Piccinelli1997,Bradley1998,Russell1994,Russell1996},
although the emphasis in these cases is on consumption at disordered
levels. 

\citet{Prior2003} considered a different health arena (mental health
problems and general practionioners), with similar characteristics in terms of concealment
of medically relevant information. The central finding in this case
is that non-disclosure is not a result of stigma, but of mismatched
ontologies surrounding mental illness. Work by \citet{Alvik2005},
where the relationship between anonymity and reporting of alcohol
consumption by pregnant women was investigated, found no significant
relationship, suggesting that a fear of social judgement may not be
a dominant factor. This draws an interesting contrast with a study
by \citet{Alvik2006}, which found that contemporaneous reports of
consumption were significantly lower than those postpartum. Logistic
regression results suggest that this trend is amplified by a number
of factors, including level of alcohol consumption preceding conception,
while anxiety about foetal well-being during pregnancy was associated
with lower retrospective reports. Taken together with \citep{Alvik2005},
these results could be seen as conflicting, but may suggest self-stigmatisation
\citep{Watson2007}, or reflect a lack of distinction between anonymity,
and confidentiality \citep{Malvin1983}.

In summation then, there is a consensus that alcohol consumption is
generally under reported in the pregnant population, with some support
for the idea that concern about social judgement associated with stigmatisation
may be a contributing factor. Of particular interest in the wider
context of this work, is the relationship between under reporting and
consumption, i.e. that heavier drinking is associated with a greater
tendency to understate intake.


\subsection{Practice Implications}

Given that alcohol consumption is thought to be under reported some
consideration must be given to the implications for midwifery practice,
in terms of eliciting more accurate self-reporting. \citet{Phillips2007}
present a qualitative account of factors influencing the disclosure
of substance misuse to midwives, identifying particularly the need
to build up a rapport, potentially over a number of appointments.
This was related to continuity of care, seen as necessary by both
midwives and women for building up a trust relationship, itself a
key component of facilitating disclosure. \citet{Stevens2002} looked
specifically at the process of transitioning to a caseloading model
of care provision in one midwifery practice, reporting that both practitioners
and women felt that this offered advantages in terms of long term
relationship building. Relationship building was also highlighted
by \citet{Kennedy2004} in a narrative investigation of midwifery
practice, where the subjects interpreted the midwife-woman dynamic
as about mutuality. \citeauthor{Kennedy2004} suggest that this arises
from a recognition that interactions in this context are about information
exchange, with the knowledge base of the woman as significant as that
of the midwife, rather than simply a one directional didactic relationship.


\citet{Hunter2006} also focuses on much of midwifery as about relationship
building, suggesting that there is an insufficiently recognised emotional
labour component to practice. Observation and interview of a number
of midwives as they practiced suggested that many midwives effectively
took a mother type role to their patients, with implications around
the nature of information exchange that was able to take place. The
emotional labour component was also reported by \citeauthor{Stevens2002},
who suggest that this is more evident under a caseloading system, particularly
with challenging patients with complex needs. \citet{Todd1998} surveyed
midwives working in a hospital environment, as well as those working
in the community in a caseloading context, finding that community
practice appeared to provide more job satisfaction, but was challenging
to implement effectively because of limited resources. Community midwives
suggested that larger team sizes, and smaller caseloads would contribute
to a better realisation of the model. \citet{Farquhar2000} approached
the same question from the perspective of women, also finding that
faulty implementation hampered the expected benefits. They found that
those cared for under a team scheme, with much higher continuity of
care, reported that they had a better relationship with their midwives,
but were not more satisfied in general with their care. In contrast,
\citet{Biro2003} looked at a \ac{RCT} of team midwifery care versus
hospital care in Australia, finding a significantly higher level of
satisfaction under the team model, the distinction in this case may
lie in the different balance of team size to caseload size.

In terms of the impact of continuity of care on health, rather than
experiential outcomes, research is relatively sparse. \citet{Marks2003}
examined the impact of continuity of care on \ac{PND}, which has
similar features to alcohol in that it carries an associated stigma
that can act as a barrier to help seeking \citep{Dennis2006}. Based
on the results of a \ac{RCT}, they conclude that continuity of care
is not protective, in the sense of reducing rates or impacting onset,
but was very successful in supporting engagement with treatment. Echoing
this, a 2009 Cochrane Review by \citet{Hatem2009} found no significant
difference in incidence of \ac{PND}, but reported benefits in terms
of lower rates of \gls{episiotomies}, anaesthesia, and shorter hospital
stays, with higher satisfaction as found by \citeauthor{Biro2003}.
While research in this area does not specifically pertain to disclosure,
the general trend in results are suggestive when taken in concert
with studies emphasising the importance of relationship building as
key in fostering a disclosure friendly environment. Continuity of
care is generally regarded as improving patient experience, and leading
to better health outcomes in the wider medical arena \citep{VanWalraven2010},
but is clearly not a cost free endeavour, with particular concern
arising from the emotional cost \citep{Todd1998}, and increased rates
of `\gls{burnout}' in practitioners \citep{Sandall1997}.