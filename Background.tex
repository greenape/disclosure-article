
\section{Background}

\label{ch:lit_review}

This chapter presents an overview of literature focusing on the impact
of drinking behaviour in pregnancy, and factors affecting disclosure
behaviour in the midwifery context. This is followed by a review of
literature supporting the theoretical underpinning of the modelling
approach, with particular reference to statistical decision theory.


\subsection{Alcohol, and Disclosure in the Maternity Setting}




\subsubsection{Impact of Alcohol\label{sub:Impact-of-Alcohol}}

Distinct from stigma attached to alcohol consumption in pregnancy,
is the question of the real impact on woman and baby both in the \gls{antenatal}
period, and beyond. While the canonical example of alcohol linked
disorders is \ac{FAS}, and others on that spectrum, heavy drinking
during pregnancy has been mooted as a factor in a variety of negative
health outcomes. 

The impact of moderate alcohol consumption in pregnancy is more contested.
For example, \citet{Andersen2012} examined moderate drinking in a
large Danish cohort study, finding a significant increase in the risk
of spontaneous abortion at low levels of consumption early in pregnancy.
\citet{Savitz2012} questioned the extent to which this can be interpreted
as a causal connection, noting that there is a known relationship
between absence of morning sickness, and spontaneous abortion, and
suggesting that this may explain much of the difference in risk. \citet{Kesmodel2002}
examined the relationship between alcohol consumption and still-birth,
finding that increased consumption lead to an increase in risk to
the baby, but in contrast to \citet{Andersen2012} this was significant
at \gls{term}. 

Considering longer term negative outcomes, a metastudy by \citet{Latino-Martel2010}
examined the potential for maternal alcohol consumption in pregnancy
to feature as a risk factor for onset of childhood leukaemia, finding
that any alcohol consumption was associated with an increased risk
of childhood \ac{AML}, but note the rarity of the condition as a
limitation.

\citet{Huizink2006} reviewed literature looking at the impact of
moderate consumption on neurodevelopmental and cognitive outcomes,
concluding that maternal consumption can be a contributing factor
to \ac{ADHD}, and impairments to learning and memory. They subsequently
suggest that the underlying mechanism is not specific to alcohol consumption,
but a more general phenomena arising from perturbations to foetal
conditions \citep{Huizink2009}, but caution that methodological issues
in many of the studies reviewed may undermine this hypothesis.

Contrary to this, a meta-study by \citet{Gray2006} found there was
insufficient evidence to suggest any harm arising from moderate (under
1.5 UK units per day) alcohol consumption. This ties to the current
guidance from the \ac{NICE} \citep{NICE2010a}, advising that women
should avoid drinking at all in at least the first three months of
pregnancy, and no more than 1-2 units once or twice a week if they
do. In giving this advice, \ac{NICE} acknowledge that the risks to
the foetus from alcohol are a somewhat contentious subject, concluding
that the evidence of harm is inconclusive, but that this is not sufficient
to rule out the risk of negative outcomes. This tension is reflected
by earlier guidance from the \ac{RCOG} \citep{Gynecologists1996}
suggesting no evidence of harm below 15 units per week, and subsequent
criticism by \citet{Guerri1999}, who suggest that this might be interpreted
as legitimising \gls{binge}, while noting several studies indicating
adverse affects linked to even a single drink per day (e.g. \citet{Day1990}).
A subsequent \ac{RCOG} statement \citep{RoyalCollegeofObstetriciansandGynaecologists2006}
revised the recommendations to incorporate newer findings, advising
that there is no known safe threshold for drinking in pregnancy, and
highlighting binge drinking as of particular concern.

There has recently been an increased interest in the impact of binge
drinking, as a distinct pattern of consumption, with a wide variety
of negative outcomes reported by \citet{Maier2001}, although a significant
portion of their evidence base is drawn from animal studies which
augers for caution in generalising findings to humans. \citet{Strandberg-larsen2008}
explored links between binge drinking, and stillbirth, reporting a
statistically significant increase in risk associated with more than
three antenatal binge episodes. \citet{Sun2009} looked at seizure
disorders in children whose mothers binged during pregnancy. They
reported significantly greater risk of both neonatal seizures (\textasciitilde{}3
fold) and epilepsy (1.81 fold) associated with binge drinking between
11 and 16 weeks, but emphasised the exploratory nature of the results,
and need for replication. In terms of neurodevelopmental outcomes,
\citet{Streissguth1994} found a dose dependent association with scores
on timed word, and arithmetic tests in fourteen year olds with a stronger
association where bingeing occurred. A review by \citet{Henderson2007}
cautiously supports the contention that binge drinking has a neurodevelopmental
impact, but found no consistent support for adverse outcomes in pregnancy
(e.g. stillbirth, miscarriage, etc.) and note a paucity of studies
in the area. \citet{Meyer-Leu2011} considered the neonatal period,
finding that both moderate and binge drinking were associated with
an increased trend towards neonatal asphyxia. They also noted a large
number of contradictory findings and raising methodological concerns
about the studies reviewed by \citeauthor{Henderson2007}. \citet{Barr2006}
contend that binge drinking may also contribute to psychiatric issues
in the later life of offspring, although in this case their findings
are confined to individuals with \ac{FAS}, which may in itself be
a confounding factor, rather than indicating a directly causative
relationship between antenatal binge drinking and subsequent psychiatric
disorder in offspring.

Overall, there is a distinct lack of consensus on what, and how extensive,
the effects of drinking on the immediate and long term health outcomes
are for the child. 


\subsubsection{Disclosure}



The issue of disclosure is central to the model presented here, in
particular self-report by women of information that might disadvantage
them, or be expected to do so in the immediate term. In general, the
consensus is that alcohol self-reports have acceptable validity in
the research context \citep{DelBoca2000}, but do not correspond perfectly
to alternative methods. \citet{DelBoca2003} claim that the validity
is generally accepted, and suggest that the current focus lies on
what factors and processes underlie the discrepancies rather than
questioning determining their existence. In this instance, the conjecture
is that the information is in some way stigmatising; that, following
\citet{Goffman1999}, disclosure equates to revelation of the mark.
This is not immediately contentious, for example \citet{Gomberg1988}
identified stigma surrounding alcohol abusing women in particular,
an issue also highlighted by \ac{IAPT} guidance \citep{IAPT2012},
as well as a number of other studies relating response effects to
perceived negative consequences \citep{Langenbucher2001,DelBoca2000,Blair1977}.
In the maternity context, \citet{Radcliffe2011} identifies stigma
pertaining to substance misusing women amongst staff, and suggests
that this may represent a barrier to appropriate treatment; similarly,
\ac{NICE} guidance on pregnancy and complex social factors \citep{NICE2010}
recognises concern about the attitude of staff as a source of anxiety
in pregnant women who misuse substances.

Stigma, or fear of a judgemental response on the part of the practitioner
should not however be taken uncritically to explain inaccurate reporting
by patients. While recent \ac{NICE} public health guidance advocates
routine alcohol misuse screening as a part of all practice \citep{NationalInstituteforHealthandCareExcellence2010},
there is no specific policy for routine antenatal care beyond providing
information on possible impacts of alcohol consumption \citep{NICE2010a}.
\ac{NICE} guidance on pregnancy and complex social factors \citep{NICE2010}
does specifically address women who misuse alcohol, but presupposes
knowledge of the problem through medical history, or via other services.
Taken in concert with the potential for harm from even moderate alcohol
use (section \ref{sub:Impact-of-Alcohol}), this suggests that much
of the onus is on the patient to volunteer information.

Where screening is used, \citet{Kaskutas2000} note that the most
basic method, i.e. number of standard drinks consumed, can lead to
inaccurate estimates of consumption arising from inability to relate
the concept of a standard drink, to actual consumption. This is compounded
by the impact of memory effects on recall over a number of days \citep{Stockwell2004},
and a lack of consistency in the standard drink measure \citep{Turner1990}.
Alternative screening tools, for example \acs{AUDIT}, and \acs{T-ACE}
are available and have been shown to perform well in identifying problematic
levels of drinking \citep{Piccinelli1997,Bradley1998,Russell1994,Russell1996},
although the emphasis in these cases is on consumption at disordered
levels. 

\citet{Prior2003} considered a different health arena (mental health
problems and GPs), with similar characteristics in terms of concealment
of medically relevant information. The central finding in this case
is that non-disclosure is not a result of stigma, but of mismatched
ontologies surrounding mental illness. Work by \citet{Alvik2005},
where the relationship between anonymity and reporting of alcohol
consumption by pregnant women was investigated, found no significant
relationship, suggesting that a fear of social judgement may not be
a dominant factor. This draws an interesting contrast with a study
by \citet{Alvik2006}, which found that contemporaneous reports of
consumption were significantly lower than those postpartum. Logistic
regression results suggest that this trend is amplified by a number
of factors, including level of alcohol consumption preceding conception,
while anxiety about foetal wellbeing during pregnancy was associated
with lower retrospective reports. Taken together with \citep{Alvik2005},
these results could be seen as conflicting, but may suggest self-stigmatisation
\citep{Watson2007}, or reflect a lack of distinction between anonymity,
and confidentiality \citep{Malvin1983}.

In summation then, there is a consensus that alcohol consumption is
generally underreported in the pregnant population, with some support
for the idea that concern about social judgement associated with stigmatisation
may be a contributing factor. Of particular interest in the wider
context of this work, is the relationship between underreporting and
consumption, i.e. that heavier drinking is associated with a greater
tendency to understate intake.


\subsubsection{Practice Implications}

Given that alcohol consumption is thought to be underreported some
consideration must be given to the implications for midwifery practice,
in terms of eliciting more accurate self-reporting. \citet{Phillips2007}
present a qualitative account of factors influencing the disclosure
of substance misuse to midwives, identifying particularly the need
to build up a rapport, potentially over a number of appointments.
This was related to continuity of care, seen as necessary by both
midwives and women for building up a trust relationship, itself a
key component of facilitating disclosure. \citet{Stevens2002} looked
specifically at the process of transitioning to a caseloading model
of care provision in one midwifery practice, reporting that both practitioners
and women felt that this offered advantages in terms of long term
relationship building. Relationship building was also highlighted
by \citet{Kennedy2004} in a narrative investigation of midwifery
practice, where the subjects interpreted the midwife-woman dynamic
as about mutuality. \citeauthor{Kennedy2004} suggest that this arises
from a recognition that interactions in this context are about information
exchange, with the knowledge base of the woman as significant as that
of the midwife, rather than simply a unidirectionally didactic relationship.


\citet{Hunter2006} also focuses on much of midwifery as about relationship
building, suggesting that there is an insufficiently recognised emotional
labour component to practice. Observation and interview of a number
of midwives as they practiced suggested that many midwives effectively
took a mother type role to their patients, with implications around
the nature of information exchange that was able to take place. The
emotional labour component was also reported by \citeauthor{Stevens2002},
who suggest that this is more evident under a caseloading system, particularly
with challenging patients with complex needs. \citet{Todd1998} surveyed
midwives working in a hospital environment, as well as those working
in the community in a caseloading context, finding that community
practice appeared to provide more job satisfaction, but was challenging
to implement effectively because of limited resources. Community midwives
suggested that larger team sizes, and smaller caseloads would contribute
to a better realisation of the model. \citet{Farquhar2000} approached
the same question from the perspective of women, also finding that
faulty implementation hampered the expected benefits. They found that
those cared for under a team scheme, with much higher continuity of
care, reported that they had a better relationship with their midwives,
but were not more satisfied in general with their care. In contrast,
\citet{Biro2003} looked at a \ac{RCT} of team midwifery care versus
hospital care in Australia, finding a significantly higher level of
satisfaction under the team model, the distinction in this case may
lie in the different balance of team size to caseload size.

In terms of the impact of continuity of care on health, rather than
experiential outcomes, research is relatively sparse. \citet{Marks2003}
examined the impact of continuity of care on \ac{PND}, which has
similar features to alcohol in that it carries an associated stigma
that can act as a barrier to help seeking \citep{Dennis2006}. Based
on the results of a \ac{RCT}, they conclude that continuity of care
is not protective, in the sense of reducing rates or impacting onset,
but was very successful in supporting engagement with treatment. Echoing
this, a 2009 Cochrane Review by \citet{Hatem2009} found no significant
difference in incidence of \ac{PND}, but reported benefits in terms
of lower rates of \gls{episiotomies}, anaesthesia, and shorter hospital
stays, with higher satisfaction as found by \citeauthor{Biro2003}.
While research in this area does not specifically pertain to disclosure,
the general trend in results are suggestive when taken in concert
with studies emphasising the importance of relationship building as
key in fostering a disclosure friendly environment. Continuity of
care is generally regarded as improving patient experience, and leading
to better health outcomes in the wider medical arena \citep{VanWalraven2010},
but is clearly not a cost free endeavour, with particular concern
arising from the emotional cost \citep{Todd1998}, and increased rates
of `\gls{burnout}' in practitioners \citep{Sandall1997}.


\subsection{Games, Signals, and Decisions}




\subsubsection{Signalling Games}



Game theory generally deals with strategic decision making in the
unusual circumstance of complete information, that is, every player
has at least complete knowledge of all possible outcomes, who their
opponents are, and so forth. Arguably more generally applicable is
the incomplete information scenario, where players lack information
about the rules of play in some fashion. \citet{Harsanyi1967} proposed
a method for effectively transforming such games into games of complete
information by treating the possible variations on the rules as subgames.
To determine which subgame is to be played, an additional player -
nature - is introduced to make the first move, where nature conducts
a lottery according to some probability distribution. If it is assumed
that the underlying probability distribution is known to all players,
the game is then one of complete information. 

Perhaps the best known example of Bayesian games, are the signalling
games codified by \citet{Kreps1987}, after initially being framed
by \citet{Spence1973} in the context of employment markets. The general
form of such a game is that one player holds information known only
to them, on the basis of which they send a signal to the other player(s),
which the other player(s) then act upon. Much of the interest in signalling
games turns on what conditions are necessary for honest signalling
to be a Nash equilibrium, or in the context of evolutionary game theory,
an \ac{ESS} . 

One approach to this requires that signalling is a costly exercise,
as proposed by \citet{Grafen1990} in examining biological signals
(for example, the eye-catching but unwieldy peacock tail). \citeauthor{Grafen1990}
demonstrated that an earlier suggestion by \citet{Zahavi1975}, who
proposed that such signals were in effect a handicap demonstrating
fitness, would lead to an \ac{ESS} because of the costly nature of
the signalling. This solution is also noted by \citet{Spence1973},
who showed that a separating equilibrium exists%
\footnote{In fact, an infinite number of them.%
} contingent on signals being more costly for some types.

Costly signalling has been applied to explain a variety of apparently
contradictory behaviours, for example \citet{Godfray1991} in the
context of offspring soliciting food from parents, where the key question
is why a behaviour with potentially very high costs (namely, being
eaten) would be preferred to a less risky method. In a social context,
costly signalling has been proposed as an explanation for religion
in human societies. \citet{Sosis2003} developed a model of religious
ritual as an exercise in costly signalling, showing that higher costs
to engagement in rituals for skeptics maintains the stability of religious
groups and the presumed benefits that membership confers. \citet{Henrich2009}
extended this idea, and developed an evolutionary model combining
cultural transmission with costly signalling in a population, finding
that for even modest costs the system moved towards universal belief.
\citet{Wildman2011} subsequently extended the model, to address the
fact that both stable equilibria are binary states, finding that the
incorporation of group differentiation allowed subgroups to persist.

Signalling games have also been extended to provide models of other
observed human behaviour, for example \citet{Austen-Smith2005} attempted
to explain the observed poor academic attainment of some social groups
by positing a multiple audience signalling game. They found that the
introduction of a secondary signalling game with a peer audience,
alongside the prototypical \citeauthor{Spence1973} model introduced
a pooling equilibrium. Subsequent empirical work by \citet{Jr2010}
has provided some support for this idea. Along similar lines, \citet{Feltovich2002}
examine an observed failure by high quality types to signal as would
be anticipated, introducing the concept of countersignaling in scenarios
where there is noisy leakage of type information. They found that
where there is added noisy type information available, separating
equilibriums exist where high quality senders signal either as low
quality, or not at all.


\subsubsection{Bayesian Decision Theory and Expected Utility}

Decision theory is the theory of rational decision making \citep{Peterson2009},
this contrasts with game theory which is concerned with strategic
decision making. In the broadest sense, the field can be divided into
two types of theories: normative, and descriptive. Normative theories
are those which attempt to give the rational answer to a decision
problem, descriptive or behavioural theories focus instead on characterising
the process of human decision making. In this instance, the particular
concern is with theories of decision making under uncertainty.

Underpinning almost all theories of decision making, and much of economic
theory in general is the concept of expected utility, originally proposed
by \citet{Bernoulli1954}. This casts decisions as choices between
lotteries or gambles, with differing payoffs and probabilities.

Under this model, the expected utility of any gamble is a function
of the probability of the outcomes, their utility to the gambler,
and the gambler's risk aversion. Essentially this is an extension
of the expected value criterion, which assumes that the expected value
is based only on the probability and objective value of outcomes.
By contrast, the utility framing is a subjective measure, allowing
differing preferences between gamblers. \citet{Neumann1953} later
formalised the theory, defining rational decision as acting to maximise
expected utility, where an individual's preferences are shown to fulfil
four axioms, namely completeness, transitivity, independence, and
continuity. Completeness requires that for any two lotteries A and
B, the decision maker prefers one to the other, or is indifferent.
Transitivity requires that if A is preferred to B, and B is preferred
to C, then A is also preferred to C. Continuity states that given
a scenario as in the transitivity axiom, there is some combination
of lotteries A and C where the decision maker is indifferent between
that combined lottery and B. Finally, independence maintains that
if one were to prefer gamble A to B, that preference holds if both
are combined with lottery C.

While vastly influential, the expected utility theory has been substantially
criticised, generally for failing to predict real behaviour. \citet{Society2013}
attacked the independence axiom in particular, suggesting that in
some scenarios people's choices would be inconsistent where expected
utility implies otherwise. A number of studies (e.g. \citep{Oliver2003,Burke1996})
have since supported the intuition to some extent.

More recently, support for some aspects of the expected utility theory,
particularly the concept of utility as a common currency for comparison,
has come from neurology, for example following work by \citet{Platt1999},
\citet{Padoa-Schioppa2006,Padoa-Schioppa2008} report neuronal firing
corresponding to economic value in decision making tasks undertaken
by monkeys, while \citet{Christopoulos2009} found similarly indicative
results for risk aversion. The suggestion implicit in the model proposed
here, that this also applies to social judgements, is less investigated,
although both \citet{Watson2012}, and \citet{Willis2010} found that
lesions in the brain area%
\footnote{The orbitofrontal cortex.%
} identified by \citeauthor{Padoa-Schioppa2006} lead to abnormal social
judgements in humans and primates.

Bayesian decision theory, as expounded by \citet{Robbins1964} applies
Bayesian inference to the process of decision making under some degree
of uncertainty, on the basis that the decision is a repeated one.
The central idea is relatively straightforward, and assumes that the
loss or gain of some action to resolve a decision is contingent on
an unknown parameter. To solve the problem, the decision maker chooses
whichever action will minimise the risk, where the risk of an action
is $\underset{i}{\sum}\lambda(a_{j}|w_{i})P(w_{i}|x)$, i.e. the loss
incurred for taking action $a_{j}$ given that the true state of the
world is $w_{i}$, multiplied by the belief that this is the true
state of world given evidence $x$, summed across all possible worlds.
Essentially this is identical with expected value, with Bayesian style
probabilities. This allows an additional process of inference to progressively
update the distribution from which $P(w_{i}|x)$ derives, as new evidence
is obtained after each decision. 

This approach has been used in a wide variety of scenarios, for example
\citet{McNamara1980} have applied statistical decision theory as
a framework for understanding animal learning%
\footnote{Although they note that this is in the sense of how animals `should'
learn, rather than how they do learn%
}, while \citet{Harsanyi1978} has derived an ethical framework from
the principles. Less controversially, in contexts where optimality
is desirable as an outcome, \citet{Survey2003} have used Bayesian
decision methods in combination with \ac{MCMC} to solve complex waterfowl
habitat management problems, and \citet{Kristensen1997} has developed
robots which utilise Bayesian decision analysis to plan sensor operations.

As with standard expected utility, the Bayesian approach can be criticised,
in this case on the grounds of plausibility. The question of plausibility
arises from the suggestion that Bayesian inference is in some way
a model of human inductive reasoning, as argued by some branches of
cognitive science. For example, \citeauthor{Tenenbaum2006} argue
for the Bayesian approach as a top-down model of inductive reasoning
in humans \citep{Tenenbaum2006,Griffiths2010}, a general approach
criticised by \citet{Bowers2012} as unfalsifiable, overcomplicated,
and relying on an unrealistic conceptualisation of the brain as optimal.
\citet{Miller2012} also applied similar criticism to claims by \citet{Gallistel2012}
that Bayesian inference better characterises learning as opposed to
associative conditioning type models, suggesting that this relies
on an assumption of optimality which is unfounded.


\subsubsection{Descriptive Decision Theory}

Arguably the most significant criticism of theories of decision making,
is their failure to correspond to empirically observed decision making
\footnote{This critique is not unique to decision theory, and has also been
levelled at game theory (e.g. \citet{Fehr2003} on the irrational
altruism of humans playing the prisoners' dilemma).%
}. This was probably first raised by \citet{Simon1956}, who proposed that the
apparent divergence derived from a tendency to satisfice, rather than
optimise. This suggestion rests on the not unreasonable assumption
that people do not have unlimited cognitive capacity (i.e. bounded
rationality \citep{Simon2000}), and hence use heuristic means to
make decisions, namely by choosing the first `good enough' option.
\citeauthor{Simon1956} suggests that this process nevertheless leads
to the optimal solution is most cases.

Subsequent work on descriptive theories largely follows the same framework
in assuming that in reality, human decision making is a heuristic
process. \citet{Tversky1974} developed three heuristics to explain
observed systematic errors in reasoning - representativeness, availability,
and anchoring. Representativeness suggests that when asked to judge
how related one object or event is to another, they do this based
on the extent to which they resemble one another - crucially they
will ignore additional, better information when available. Availability
claims that when tasked with estimating probabilities, people will
rely on the ease with which they can call examples to mind (note that
this might be considered an example of satisficing). Finally, anchoring
proposes that when estimating, people start with some initial value
and progressively update from there, i.e. they will tend to overweight
prior evidence at the expense of new information. 

Subsequently, \citet{Kahneman1984,Tversky1986} also identified framing
effects, which imply that the decisions people make are impacted by
the fashion in which the problem is presented. The essential outcome
from these findings is that people are risk seeking when faced with
outcomes framed as losses, but risk averse towards gains, and regard
any loss as greater than an equivalent gain. The impact of framing
in itself has been shown to be significant, for example \citet{Toll2007}
found improved abstinence rates in smoking cessation where quitting
was framed as a gain, and \ac{NICE} recommend considering the framing
of treatment outcomes when presenting options to patients \citep{NICE2007}.

\ac{PT} \citep{Kahneman1979} attempts to provide a decision rule
accounting for the heuristic nature of decision making and incorporate
framing effects, which successfully explains many perceived failures
of rationality. A revised version, \ac{CPT} \citep{Tversky1992}
addressing a violation of first order stochastic dominance possible
under the original formulation, extends the theory to allow decisions
with more than two options, but sacrifices the editing phase. \citet{Camerer2004a}
reviews a number of successes in explaining apparent anomalies with
\ac{CPT}, and argues that should replace expected utility in general
usage. \citet{Thaler2000} regards the theory as promising, but points
out that it is in many ways incomplete, citing the lack of explanation
as to how people construct frames as an example of this.

A significant weakness of \ac{CPT} as a general theory of decision
making is that it fails to account for behaviour under intertemporal
choice, or rather does not attempt to address it. Generally, intertemporal
choice is assumed to be underpinned by the \ac{DU} model of \citet{Samuelson1937},
which proposes that the value of a thing right now is greater than
the value of it at some point in the future (jam today has more utility
than jam tomorrow), following an exponential relationship. A more
nuanced view of this has been proposed by \citet{Ainslie1991}, suggesting
that the relationship is hyperbolic rather than exponential. Both
models however fail to explain several inconsistencies, for example
\citet{Thaler1981} found that discounting rates were different between
gains and losses. \citet{Loewenstein1992} report a number of additional
inconsistencies that are not adequately resolved by \ac{DU} models,
and propose an alternative along the lines of \ac{CPT} to resolve
them while retaining the capabilities of Kahneman and Tversky's model
in immediate term choices.
