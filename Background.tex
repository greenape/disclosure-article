%!TEX root = disclosure_game.tex
%Rewrite all of this, ditch the alcohol stuff altogether. Any really key can be woven in elsewhere (discussion)
\section{Previous Research}

\label{sec:lit_review}

This section presents a brief overview of previous work relating to the theoretical backdrop to this approach, addressing in turn signalling games, normative decision theory, heuristic decision making, and descriptive decision theory.


\subsection{Signalling Games}

The majority of classical game theory focuses on strategic decision making, in scenarios where all players have complete information about all aspects of the game and their opponents. Moves by players are assumed to be perfectly rational, and account for the equally rational moves of their opponents.
An alternative, perhaps more common situation, is that players have incomplete information, whether about the actions available to their opponents or themselves, or concerning the possible payoffs for the players.   
This second case was essentially unexplored, until \citet{Harsanyi1967} introduced the concept of a Bayesian game, which treats incomplete information scenarios by allowing the possible variations on the rules to be treated as subgames. This adds an additional player - nature, to the game, where nature takes the first move thereby deciding which subgame is played. Nature is assumed to make their move by lottery, and where the probability distribution governing the lottery is known to all players this permits the game to be formulated as one of complete information. 

Here, we are specifically interested in signalling games \citep{Spence1973,Kreps1987}, a type of Bayesian game where one player holds some private information which may be communicated (or not) by means of a signal.
This basic form has been widely applied, with substantial interest in what conditions permit honest signalling as Nash equilibria or \acp{ESS}. \citet{Grafen1990}, following from a suggestion by \citet{Zahavi1975}, proposed that if signals intended to indicate mate quality exacted a cost on the signaller (e.g. peacock tail feathers), then honest signalling would constitute an \ac{ESS}. Similar results have also been demonstrated in a game of job market signalling, where signal cost was differentiated by type \citep{Spence1973}. 
Costly signalling has also been suggested as an explanation of behaviour that at first glance appears counter intuitive, for example \citet{Godfray1991} applied the idea to the food solicitation behaviour of chicks, where a stronger signal (i.e. more, or louder chirping) carries a risk of being eaten by predators. Moving beyond animal behaviour, \citet{Sosis2003} considered the implications of ritual behaviour, in the context of religion, representing a costly signal, an idea subsequently extended by \citet{Henrich2009} to include cultural transmission, and \citet{Wildman2011} to introduce group differentiation.

Other work augments the signalling game model, for example \citet{Austen-Smith2005} add a second `peer group' audience signalling game to the original \citeauthor{Spence1973} game in an effort to explain poor academic performance in some social groups, with some subsequent empirical support for the idea from \citet{Jr2010}. On a similar tack, \citet{Feltovich2002} introduce additional noisy type information, finding that this effectively explained counter-intuitive observed behaviour where actors with every right to boast of their quality fail to do so.

\subsection{Normative Decision Theory}

While game theory addresses rational \emph{strategic} decision making, decision theory deals instead with rational decision making \citep{Peterson2009}.  Taken literally, this leads to normative decision theory, where the focus is on giving the rational answer to a decision problem. A complementary view - descriptive, or behavioural decision theory, holds that the focus should instead be on giving an account of human decision making performance, complete with observed deviations from perfect rationality, which we address in section \ref{sub:descriptive_theories}. Finally a third perspective, which to some extent overlaps this division, suggests that decisions are heuristic in nature and rational in ecological context \citep{Gigerenzer1996} (section \ref{sub:heuristic_theories}).

The conceptual underpinning of all of these is the central idea of expected utility, originated by \citet{Bernoulli1954} and later formalised by \citet{Neumann1953}. Utility in this sense is the subjective value of an outcome, often expressed in monetary terms, and expected utility incorporates the uncertainty of obtaining that outcome. This is expressed by taking the expected utility of an outcome to be the probability of the outcome occurring, multiplied by the subjective value of it.

Recently, several studies have explored biological correlates to aspects of expected utility. The fundamental concept, that all outcomes are comparable in a universal currency has been supported by evidence of neural correlates of decision variables \citep{Platt1999}, and following from this results from \citet{Padoa-Schioppa2006,Padoa-Schioppa2008} showing neuronal firing in the \ac{OFC} corresponding to revealed preferences in monkeys. Additionally, some support for neural representation of value, and risk aversion was found by \citet{Christopoulos2009}. The model presented in this paper makes an explicit assumption that social decisions utilise the same process, and while this is less well supported there is some evidence to suggest involvement by the same brain region, since damage to the \ac{OFC} has been shown to impair social judgements in both primates \citep{Watson2012}, and humans \citep{Willis2010}.

An alternative normative model of decision making is Bayesian decision theory, proposed by \citet{Robbins1964}, which is essentially the application of Bayesian style probabilities to the expected utility model. This allows probabilities used in reasoning to be subjective, which may allow for a better account of decisions from experience (see \citet{Hertwig2004,Hau2008} for results elucidating the distinction, and comparing the performance of several non-Bayesian models). This model has seen notable successes in practical problems \citep{McNamara1980,Kristensen1997,Survey2003}, but suggestions by several authors  that it could constitute an effective (top-down) model of learning \citep{Tenenbaum2006,Griffiths2010}, or induction \citep{Gallistel2012} in the brain have attracted substantial criticism\footnote{For example \citet{Bowers2012} responding to \citet{Tenenbaum2006}, and \citet{Griffiths2010}; and \citet{Miller2012} addressing \citet{Gallistel2012}.}.

\subsection{Heuristic Decision Making}\label{sub:heuristic_theories}

As noted, heuristic decision making stems from a contention that \citeauthor{Neumann1953} type rationality ignores the context of decision making, and a lack of correspondence between predicted and actual human decisions (see, for example the Allais paradox \citep{Society2013}, and subsequent empirical support from \citet{Burke1996} and \citet{Oliver2003}). Arguably, this notion begins with \citet{Simon1956}, who suggested that humans do not attempt to make optimal choices, but to satisfice and choose the first `good enough' option. While noting that this will often achieve the same result, the claim is that humans exhibit bounded rationality \citep{Simon2000} arising from inherent limits to cognition.

\citet{Gigerenzer1996} take the concept of bounded rationality further, and argue for what they term \acp{FFH}. This recasts rationality as bound to the context of the behaviour: a rational approach to choosing the right mate might well require checking every possible partner, but given finite time, memory, and so on rapidly becomes unachievable. On this basis, they contend that the rationality of any given decision rule can only be determined in the context of the environment \citep{Todd2003}, which implies that heuristics are task specific. They provide a number of heuristics for varying decision problems, e.g. recognition \citep{Goldstein2002}, cue ordering \citep{Gigerenzer1999,Todd2004}, and binary decisions \citep{Brandstatter2006}.

\subsection{Descriptive Decision Theory}\label{sub:descriptive_theories}

While heuristic theories arguably fall under the purview of the descriptive, the wider tendency is towards what are in essence \enquote{patches} to normative models. The most influential models in this class derive from \ac{PT} \citep{Kahneman1979}, which combines a set of heuristics based on observed decision behaviour \citep{Tversky1974}, with distortions to the perception of probability, and the value of outcomes \citep{Kahneman1984,Tversky1986}. \citet{Tversky1992} subsequently addressed issues present in their original formulation by introducing \ac{CPT}, which allows for non-binary decisions, but dispenses with the heuristic aspects of the original formulation. The essence then, is that high and low probabilities are treated differently, and the subjective value of a loss differs from the equivalent gain (losing your shirt is perceived as more of a loss than winning a shirt is a gain).
This last effect, known as the framing effect is particularly significant, see for example work by \citet{Toll2007} examining the relationship between loss and gain framings and success rates in giving up smoking, and \ac{NICE} guidance on framing of treatment options \citep{NICE2007}.

\ac{CPT} has been successful in explaining a number of anomalous results in decision tasks (see \citet{Camerer2004a} for a review), and \citet{Thaler2000} comments to the effect that the theory is promising, albeit incomplete, lacking for example any explanation of how frames are constructed. While remaining an effective account of decision behaviour under risk, the theory does not attempt to resolve apparent inconsistencies that arise when outcomes are delayed, i.e. in situations of intertemporal choice. Historically, \ac{DU} \citep{Samuelson1937}, which effectively claims that the value of a thing now is exponentially greater than the promise of the same thing at some future date, has been applied to explain this. More recently, \citet{Ainslie1991} has suggested that discounting of future outcomes is hyperbolic, rather than exponential. However neither model is complete, in that both fail to account for results from \citet{Thaler1981} showing differing temporal discounting rates for losses and gains. \citet{Loewenstein1992} report additional failings in classic \ac{DU} models, and propose a modified form of \ac{CPT} which they suggest is able to handle both immediate, and intertemporal choice.