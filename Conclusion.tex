%!TEX root = disclosure_game.tex
\section{Discussion and Conclusions}
\label{sec:conclusion}
%reiterate contribution
%	bring up the simplicity -> chaos thing, talk about homophily
%	talk up the good (trends! yay!) and intro the bad -> no data
%
%	emphasise the 'nice story' -> future work
% ~ 1,500 words.

From a pragmatic perspective, the differing response characteristics of the decision rules are substantial and significant. The lack of a mechanism of homophily in the very simple rules leads to a high level of uncertainty in the overall dynamics, since even irrelevant information is taken at face value and to be equally valuable. By contrast, the belief structure of the \ac{CPT}, and Bayesian models represents a useful abstraction which allows the use of the relevant aspects of the information.  Naturally, incorporating homophily, by, for example weighting shared information by the type of the sharer, would represent a relatively trivial modification to the heuristic models. While to some extent this highlights the flexibility of the decision rule approach, it would of course sacrifice the parsimony of the model to a degree. This is an important consideration, given that part of the argument in favour of a decision theoretic approach to agent building lies in the minimal nature of the behavioural rules, which can be seen as occupying a middle ground between parameter and model.

One of the notable features of the results is the similarity in behaviour of the two classes of decision rule. To some extent this reflects poorly on the most complex rule, \ac{CPT}, which diverges only relatively minorly in behaviour from the Bayesian model. To some extent this might be anticipated, in that the payoffs are clearly unrealistic, which limits the utility of the \ac{CPT} approach.  Additionally, work by \citet{Glockner2012} has shown that there is considerable variation in individual parameters for the decision model, whereas we have let them remain homogenous here. In the vein, utility functions should arguably vary between individual agents, which could potentially be addressed by replacing the fixed payoffs used here with a distribution.  With this said, the significant increase in complexity, which entails both additional parameters and increases to simulation time may auger for a middle ground, particularly where elicitation of payoffs is impractical.  This, together with the variablility associated with the heuristic type decision rules speaks to a tradeoff between capturing reality, and replicating it.

Continuing the discussion of the issues raised by the representation of payoffs, the temporal aspect is significant, in that there is clearly a timing difference in payoffs, since while the potential social pain of disclosure is immediate, the health outcome comes only later. In light of this, that there is a known impact of time on perceived utility \citet{Thaler1981} suggests that incorporating some form of temporal discounting (e.g. exponential \citep{Samuelson1937}, or hyperbolic \citep{Ainslie1991}), or a decision model which explicitly treats intertemporal choice, such as the \ac{CPT} like model of \citet{Loewenstein1992}, is warranted. 

More broadly, the results demonstrate the logistical feasibility, and utility as a `tool for thinking', of an agent model grounded in decision theory. The results also make clear that deciding the operationalisation of the decision making is of key significance.

The conclusions that can be drawn about the behaviours of real life women, and their midwives, are necessarily limited by the paucity of data available to validate the model. While qualitative trends offer a some indication, they are clearly very limited in scope, and do not permit strong claims about the drivers of disclosure, and auger for a sceptical eye as to any recommendations targetted at improving outcomes.  With this said, the trends reported by \citet{Alvik2006}, and \citet{Phillips2007} are bourne out by the model, and predictions from the two more complex rules suggest that encouraging information sharing between women may encourage disclosure, but at the expense of reducing accuracy.

Further work will focus on applying the model to domains where validation data is more available, which will support a more comprehensive evaluation of the model discrepancy.  Disclosure is a widely applicable issue in health and social care and beyond, with examples ranging from a lack of help seeking behaviour noted in older men \citep{Smith2007a}, to the inverse scenario of innapropriate disclosure in social media \citep{christofides2009information}.