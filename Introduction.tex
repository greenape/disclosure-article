%!TEX root = disclosure_game.tex
\section{Introduction}
\label{sec:intro}

The case in favour of \ac{ABM} as a general analytical approach has been made numerously, and elegantly (e.g. \cite{epstein1994growing,Resnick,Axelrod1997,gilbert1999simulation,Macy2002a,Silverman2011,Silverman2013,epstein2014agent_zero}, amongst others). As such we will not belabour the point, and instead turn to addressing some of the concerns expressed about the approach. In this instance we focus on the perception of \ac{ABM} as ad hoc in nature, reflecting the assumptions of the modeller rather than empirically or theoretically grounded \citep{Waldherr2013}. To ameliorate this concern, we draw on decision theory to produce simple rule based, learning, decision making agents and show that they are able to play a form of signalling game\footnote{In a signalling game, one player (the signaller), has some piece of information that is known only to them which affects the outcome of the game for both players. The signaller has a choice as to what they tell the other player about this hidden information, and the responding player as to what they believe the information to be.} \citep{Kreps1987} with a basic form of intragroup social learning. Four decision models of varying complexity, and behavioural plausibility are contrasted, by way of demonstrating the significance of the operationalisation of decision making in \ac{ABM}.

This exercise is framed in the context of disclosure decisions, taking drinking patterns in pregnant women as a motivating example. Alcohol consumption in the antenatal period is a significant issue in itself, although there is not a clear consensus on associated risk. In terms of official guidance in the UK, \ac{NICE} acknowledge that evidence of harm to the fetus is less than conclusive, but advise not drinking at all, or significant moderation \citep{NICE2010a}, with similar advice from the UK \cite{DepartmentofHealth2008}.

Turning more specifically to disclosure of alcohol use by women to healthcare professionals during their pregnancy, research is relatively sparse, although qualitative trends are reported by \citet{Phillips2007} and \citet{Alvik2006}. The former explored factors impacting disclosure through a small case study, highlighting the need to build up rapport between woman and midwife over several appointments; the latter compared post partum reports of consumption with contemporaneous accounts, finding apparent underreporting during pregnancy which was amplified by increased drinking. The simulation model described in this paper is able to replicate both qualitative trends, i.e. an increase in disclosure over appointments, and more honest behaviour by moderate as compared to heavier drinkers.

The resulting scenario is of substantial independent interest, and shows the potential utility of a simulation approach in domains where the process is obscured, here both because of the interest in concealment, and obvious ethical concerns. With this said, the lack of a strong quantitative evidence base against which to validate the behaviour of the model augers for caution in interpreting the results, and a necessary reminder that in this instance the model is primarily a tool for formalisation of the thought process \citep{epstein2008}, rather than a machine for predicting.


A game theoretic approach to generating an abstract form of the problem gives a convenient, and well known framework to reason about the processes involved in the scenario. While scenarios may map to a number of games, exploring one candidate game still allows for a principled comparison between interpretations, and enforces explicit assumptions. But equilibrium is the sine qua non of game theory, which is concerned with the stable outcome of an infinite contest of second guesses.
We wish to see the system in motion rather than just at rest, even if it does eventually settle to some stable point. Instead, we choose to focus instead on the behavioural processes driving a system in motion, a system out of equilibrium, to understand how these processes interact with the movement.
Introducing decision theory takes a step down the ladder of abstraction from the mental chess of game theory. Dealing instead in the mechanics of decision making, and the calculus of choice, allows us to explore not only paths that arrive at the destinations we might consider in game theory, but also avenues not accessible where we constrain ourselves to a sometimes implausible degree of rationality.

This does not preclude a strategic dimension, since decision rules are to a great extent modular, and as demonstrated in this paper can be exchanged without altering the decision problem. In addition, rules are agnostic as to the source of information, suggesting room for multi-stage processes - for example, a more game theoretic, model of the opponent's mind, type approach could act as an information source for a decision rule.  As a corollary, the decision problem agents attempt to answer can change, allowing behaviour in novel problems to be informed by beliefs derived under other conditions. This is also indicative of the broader benefits to \ac{ABM} as an approach. Embedding these abstract rules in a simulated environment allows for mechanics which cannot be readily explored using purely analytic, or predictive approaches, for example, the social learning dynamic of the disclosure game model.

While there is no universal theory of human behaviour to sit at the centre of \ac{ABM} as a method, a key motivation for decision rules is their claim to provide an account of decision making that is behaviourally and cognitively plausible. Their mooted capability in this regard is to some extent supported by work from neuroeconomics, which aims to empirically test theories of decision making \citep{Rustichini2009}. Many key aspects common to decision rules, for example the idea that a common currency is used by the brain to compare outcomes \citep{Padoa-Schioppa2006,Padoa-Schioppa2008}, are supported by neurological findings. In addition, a single decision rule represents a parsimonious alternative to numerous case specific production rules. 

Given these features, the application of decision, and game theory to \ac{ABM} is an attractive approach to computational social science, where the locus of interest is process, and decision making. Taking a balance between models focused on replication of low level neurological mechanics, and those with a higher level emphasis where individual behaviours are abstracted away, yields a computationally tractable approach. Despite the relative simplicity, it nonetheless captures some of the nuance and sophistication of human decisions.


The remainder of this paper proceeds to review the substantive context of alcohol use and disclosure in the maternity context (section \ref{sec:alcohol_disclosure}), then outlines the proposed approach to model development (section \ref{sec:midwives_model}), and experiments (section \ref{sec:method}), with selected results (section \ref{sec:results}), followed by a discussion contrasting the decision models (section \ref{sec:discussion}), and conclusions (section \ref{sec:conclusion}).