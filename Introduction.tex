%!TEX root = disclosure_game.tex
\section{Introduction}
\label{sec:intro}
%CONTRIBUTION, FUCKER!
%Also, remember to say what you're going to do for the rest of the article?
%CONTRIBUTION
%	Agents are awesome
%	Decision theory is super awesome
%	'nice story to hang it on'
% 	mention the importance of behaviour change
% tool for thinking
% theoretical vacuum at the middle
% under what scenarios does an as if explanation break down

The case in favour of \ac{ABM} as a general approach has been made numerously, and elegantly \citep{Silverman2011,Silverman2013,Resnick,gilbert1999simulation,epstein1994growing,Macy2002a,Axelrod1997}. As such we will not belabour the point, and instead turn to addressing some of the concerns expressed about the method. In this instance we focus on the perception of \ac{ABM} as ad hoc in nature, tending to be a reflection of the assumptions of the modeller rather than empirically or theoretically grounded \citep{Waldherr2013}. To ameliorate this concern, we extend previous work by \cite{GrayDissert} drawing on decision theory to produce simple rule based, learning, decision making agents and show that they are able to play a form of signalling game \citep{Kreps1987} with a basic form of intragroup information sharing. Four decision models of varying complexity, and behavioural plausibility are contrasted, by way of demonstrating the significance of the operationalisation of decision making in \ac{ABM}.

This is framed in the context of disclosure decisions, and a scenario examining drinking patterns in pregnant women which is presented in the spirit of a motivating example, rather than claimed as an accurate representation of reality. Alcohol consumption in the ante natal period is a significant issue in itself, and has been associated with many potentially negative consequences. For example, \citet{Andersen2012} report results from a large scale Danish cohort study suggesting that even low levels of consumption in early pregnancy increase the risk of spontaneous abortion, although \citet{Savitz2012} has suggested this may be attributable to a previously known link to absence of mornining sickness. Risk continues to the point of birth - \citet{Kesmodel2002} found a heightened risk to the infant - into childhood, with a metastudy by \citet{Latino-Martel2010} finding evidence on an increased risk of childhood \ac{AML} (although they suggest that the rarity of the condition is a limitation). Harm may also extend even further, and a review by \citet{Huizink2006} concluded that maternal alcohol consumption could be a contributing factor to \ac{ADHD}, and other learning impairments, but note methodological issues in a number of the papers. There is not, however, a clear consensus, with \citet{Gray2006} finding no evidence of harm below 1.5 UK units per day. In terms of official guidance, \ac{NICE} acknowledge that evidence of harm to the fetus is less than conclusive, but advise not drinking at all, or significant moderation \citep{NICE2010a}, with similar advice from the \cite{DepartmentofHealth2008}.

Turning more specifically to disclosure of alcohol use during pregnancy, research is relatively sparce, although qualitative trends are reported by \citet{Phillips2007}, and \citet{Alvik2006}. The former explored factors impacting disclosure through a small case study, highlighting the need to build up rapport over several appointments; the latter compared post partum reports of consumption with contemporaneous accounts, finding apparent underreporting during pregnancy which was amplified by increased drinking. The simulation model described in this paper is able to replicate both qualitative trends, i.e. an increase in disclosure over appointments, and more honest behaviour by moderate as compared to heavier drinkers.

This scenario is of substantial independent interest, and shows the potential utility of a simulation approach in arenas where \acp{RCT} are not viable for ethical, or financial reasons. With this said, the lack of a strong quantitative evidence base against which to validate the behaviour of the model augers for caution in interpreting the results, and a necessary reminder that in this instance the model is primarily a tool for thinking, rather than a machine for predicting.

%two fold appeal. cognitive plausibility - neuroec. inaccessible systems (litigation, concealment). fast. cheap. portable. theoretical grounding. rigor.

A game theoretic approach to generating an abstract form of the problem gives a convenient, and well known framework to reason about the processes involved in the scenario. While scenarios may well map to a plurality of games, this still allows for a principled comparison between interpretations and enforces explicit assumptions. Relating this to decision theory shifts the emphasis away from analytical equilibrium-seeking, and heightens the importance of behaviour change. This is clearly desirable where the focus is on the behavioural processes driving a system in motion, and how they change in response to that movement.
Fundamentally, the shift of emphasis is from the process of acquiring and inferring the information needed to make choices, to the process of decision.
Naturally, this does not preclude the incorporation of strategic refinement, since decision rules are to a great extent modular, and as demonstrated in this paper can be exchanged without altering the underpinning decision problem. In addition, rules are agnostic as to where the information used derives from, suggesting room for multi-stage processes.  As a corollary, the decision problem agents attempt to answer can change, allowing agents' behaviour in novel problems to be informed by beliefs derived under other conditions. 

While there is no universal theory of human behaviour to sit at the center of \ac{ABM} as a method,a key motivation for decision rules is their claim to provide an account of decision making that is behaviourally and cognitively plausible. Their mooted capability in this regard is to some extent supported by work from neuroeconomics, which aims to empirically test theories of decision making \citep{Rustichini2009}. Many key aspects common to decision rules, for example the idea that a common currency is used by the brain to compare outcomes \citep{Padoa-Schioppa2006,Padoa-Schioppa2008}, are supported by neurological findings. In addition, decision rules represent a pleasingly parsimonious alternative to explicit behavioural rules covering all possible eventualities. 

Given these features, the application of decision, and game theory to \ac{ABM} is an attractive approach to computational social science, where the locus of interest is decision making. Taking a balance between the strongly biological, e.g. neural networks, and the more abstract threshold, or microsimulation like models yields a computationally tractable approach. Despite the relative simplicity, it nonetheless captures some of the nuance and sophistication of human decisions.

%Finishing up

The remainder of this paper proceeds to provide a brief review of the methodological context (\ref{sec:lit_review}), before outlining the model (\ref{sec:model}), and experiments (\ref{sec:method}), with selected results (\ref{sec:results}), then closing with a discussion contrasting the decision models (\ref{sec:conclusion}).