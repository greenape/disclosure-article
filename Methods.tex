%!TEX root = disclosure_game.tex
\section{Method}
\label{sec:method}

This section provides details of experiments conducted to examine the ability of the model to reproduce qualitative trends reported in the midwifery literature by \citet{Alvik2006}, and \citet{Phillips2007}; as well as a global sensitivity analysis and construction of statistical emulators to explore, and contrast the response surfaces of the four decision rules.

\subsection{Qualitative Trends}
\label{sub:qt}

Throughout this paper, parameters for the \ac{CPT} model were as used in \cite{Tversky1992} (table \ref{tab:qt_params}). While there has been significant work on determining appropriate parametrisation for the model (e.g. \cite{Neilson2002,Nilsson2011,Glockner2012}, and particularly \citet{Byrnes1999} and \citet{Booij2009} addressing risk aversion and gender), a full exploration of the impact of these parameters, or heterogeneous values within populations is beyond the scope of this work. For simplicity, it was assumed that all three drinking types are equally prevalent within the population, although results derived from \acl{ALSPAC} suggest that the reality is far more positive\footnote{95.5\% of women in the sample reported consumption at, or below, \ac{NICE} recommended safe levels.} \citep{Humphriss2013}. The scenario was biased towards disclosure as the better option by presuming a distribution of midwives strongly skewed towards non-judgemental types, with beliefs initially favouring honesty. Payoffs were as in table \ref{tab:Payoff-matrix}, which ensure that it is always strictly preferable to refer drinkers, and together with the initial belief that signals will be honest, not refer those claiming otherwise.

Two key measures were used: the fraction of the subpopulation who had ever signalled honestly, and the proportion of the population who were referred. Both measures were taken after every round of play, and were taken relative to the agent's position in their sequence of appointments giving the probability of signalling honestly, or being referred having had a given number of appointments.

\begin{table}[h!]
\caption{Model parameters.} 
\label{tab:qt_params}
\begin{tabular} {lll}
\hline\noalign{\smallskip}
Name & Description & Value \\
\noalign{\smallskip}\svhline\noalign{\smallskip}
\(n_{w}\) & Number of women & 1000 \\ 
\(n_{m}\) & Number of midwives & 100 \\ 
\(r_{m}\) & Number of appointments per midwife & 1000 \\ 
\(r_{w}\) & Maximum number of appointments per woman & 12 \\ 
Runs & Simulation runs & 1000 \\ 
\(p_{w}(h)\) & Proportion of heavy drinkers & \(1/3\) \\ 
\(p_{w}(m)\) & Proportion of moderate drinkers & \(1/3\) \\ 
\(p_{w}(l)\) & Proportion of light drinkers & \(1/3\) \\ 
\(p_{m}(h)\) & Proportion of harsh midwives & \(5/100\) \\ 
\(p_{m}(m)\) & Proportion of moderate midwives & \(10/100\) \\ 
\(p_{m}(l)\) & Proportion of non-judgemental midwives & \(85/100\) \\ 
\(q_{w}\) & Probability of women sharing & 0. \\ 
\(w_{w}\) & Weight of shared information for women & 0. \\ 
\(q_{m}\) & Probability of midwives sharing & 0. \\ 
\(w_{m}\) & Weight of shared information for midwives & 0. \\ 
\(s_{i}[a_{i}]:s_{i}[a_{\neg i}]\) & Pseudo-count favouring honesty & 10:1 \\ 
\(\gamma\) & Probability weighting for gains  & 0.61 \\ 
\(\delta\) & Probability weighting for losses &  0.69\\ 
\(\alpha\) & Power for gains  & 0.88 \\ 
\(\beta\) & Power for losses & 0.88 \\ 
\(\lambda\) & Loss aversion &  2.25 \\
\noalign{\smallskip}\hline\noalign{\smallskip}
\end{tabular}
\end{table}


In addition to assessing the adequacy of the rules in capturing qualitative trends, we also examined the impact of simple social learning within the population of women (section \ref{sub:info_sharing}) on the robustness of these trends. The original experiment was repeated at \(q_{w},w_{w}\in \{0.25, 0.5, 0.75, 1 | q_{w}=w_{w}\}\), with 100 runs under each condition. 

\subsection{Global Sensitivity Analysis}
\label{sub:sensitivity}

In general, we have followed the example of \cite{Bijak2013b} for global sensitivity analysis of stochastic agent based models, although see \citet{Thiele2014} for a review of alternative techniques. For this purpose the \ac{GEM-SA} software \citep{Kennedy} was used, which implements the \ac{BACCO} method developed by \citeauthor{Oakley2004} \citep{Oakley2002,Oakley2004,Dorresteijn2010}. This is a form of variance-based sensitivity analysis, which assumes that the model output is an unknown, smooth function of the inputs. The unknown function can then be approximated as a Gaussian Process, which is fitted to the training data using Bayes' Theorem and then serves as an emulator for the simulator. The emulator is then able to provide an indication about the extent to which uncertainty in a parameter propagates to uncertainty in the output, and how sharply the output responds to change in each parameter.

\begin{comment}

Justification for doing SA, point out the wide variety of places this gets used. Talk about uncertainty, briefly raise model discrepancy.

\end{comment}

Parameters for training were generated in R \citep{RTeam2014} using and appropriately transformed Latin Hypercube Sample \citep{Carnell2012} over the space of parameters given in table \ref{tab:sa_params}, giving eleven free parameters which were treated as uniformly distributed in the range given.  Given the limitation of 400 design points for the \ac{GEM-SA} software, we produced exactly that many parameter combinations and collected results for 100 runs of each, with emulator quality assessed by leave-one-out cross validation. A fixed set of 100 random seeds was used\footnote{Fixed random seeds were used to allow simulation results to be reproducible, since the combination of a parameter set and a random seed yields a deterministic process.}, such that each parameter set was run once with each seed, for every decision rule.

To capture the response characteristics for the model, we measured four outcome variables - (1) the \ac{IQR} of the average signal sent by each type of agent in a run, (2) the average signal of moderate drinking agents in a run, and (3, 4) the \ac{IQR} of 1 \& 2 between simulation runs. Together these four metrics give an indication of how far women are separable by their signalling behaviour (1), the behaviour of the at risk drinking groups\footnote{Under most conditions, the behaviour of heavy drinkers tracks closely with their moderate counterparts.} (2), and finally the variability in the system in response to changes to the parameters (3 \& 4).

Measurements were taken at the end of 1000 rounds of play, and emulators were built against the median, and \ac{IQR} of each measure to assess both the overall trend, and the extent to which the parameters contribute to variance between runs.

Sixteen emulators were built, covering each of the four outputs on all decision models and used to conduct a probabilistic sensitivity analysis to assess the impact of parameters and interactions.

In addition to the sensitivity analysis, we also employed the resulting emulators to rapidly\footnote{Once constructed, the emulator has an analytical solution conditional on the roughness parameters, which obviates the need to use MCMC.} explore the parameter space. While emulated results are subject to inaccuracy, they do provide an indication of the interest, and plausibility, of regions of the parameter space. Results for the outcomes of the interactions of \(s_{i}[a_{i}]:s_{i}[a_{\neg i}]\) with \(x_{h}\), and \(q_{w}\) with \(w_{w}\) are given in section \ref{sub:sa_results}.

\begin{table}
\caption{Parameter ranges.}
\label{tab:sa_params}
\begin{tabular} {llll}
\hline\noalign{\smallskip}
Name & Description & Min & Max \\
\noalign{\smallskip}\svhline\noalign{\smallskip}
\(p_{w}(m)\) & Proportion of moderate drinkers & 0 & 1 \\ 
\(p_{w}(l)\) & Proportion of light drinkers & 0 & 1 \\ 
\(p_{m}(m)\) & Proportion of moderate midwives & 0 & 1 \\ 
\(p_{m}(l)\) & Proportion of non-judgemental midwives & 0 & 1 \\ 
\(q_{w}\) & Probability of women sharing & 0 & 1 \\ 
\(w_{w}\) & Weight of shared information for women & 0 & 1 \\ 
\(q_{m}\) & Probability of midwives sharing & 0 & 1 \\ 
\(w_{m}\) & Weight of shared information for midwives & 0 & 1 \\ 
\(x_{h}\) & Health payoff for healthy delivery & 1 & 100 \\ 
\(x_{r}\) & Cost for referral & \multicolumn{2}{l}{\(-(x_{h} - 1)\)} \\ 
\(s_{i}[a_{i}]:s_{i}[a_{\neg i}]\) & Pseudo-count favouring honesty & 1:1 & 100:1 \\
\noalign{\smallskip}\hline\noalign{\smallskip}
\end{tabular}
\end{table}