%!TEX root = disclosure_game.tex
\section{Disclosure Game Model}
\label{sec:model}

In this section we sketch\footnote{A complete example of this for the alcohol misuse in pregnancy model is given in appendix \ref{app:model_description}, with a schedule of simulation provided in appendix \ref{app:sim_schedule}} the process of moving from a real world scenario to a minimal game which sufficiently captures reality, expressing the result as a decision problem representation, and translating this to a simulation model. We then outline four possible decision rules, and as an example of additional flexibility of process models and simulation in contrast to purely predictive or analytical approaches, extend the model to allow a simple form of social learning.


\subsection{Modelling Approach}
\label{sec:model_design}

To model a scenario, we take the approach of first creating a formal game to represent it, capturing the key features as far as possible in the structure of that game. This game is in essence a conjecture about the real data generating process, which can be played out in simulation.

The appropriate game representative of the scenario of interest, which captures the desired strategic dynamics may not be immediately obvious. We suggest that an iterative process is beneficial, beginning from the simplest possible game, and progressively augmenting it.

Transitioning from the resulting game, to a set of decision problems is a relatively simple task. We treat the $n$ player game as $n$ one-player games \citep{RiosInsua2009}, where the moves of other players are drawn from a probability distribution - nature, in game theoretic parlance. As with the game, the decision problem representation admits a degree of variation, and may need to be adjusted to reflect the decision rules that will be used.

These decision problems may then form the basis of an agent model, where agents use learning, and decision rules to play out the game. Simulation can then support features which are not readily representable within an analytic framework, for example, populations of heterogeneous players, individual and social learning, or network effects. In addition, the ability to observe the system in a state of flux rather than at equilibrium is desirable, since even where a social system reaches a stable state, the process by which we arrive at it is significant. 

\subsection{Scenario}
\label{sec:scenario}

Typically in the UK, women have 12 appointments with a midwife during the antenatal period, and in the majority of cases will encounter several different midwives \citep{Redshaw2014} in the course of their care. In the UK, and unlike most healthcare contexts, maternity notes are held by the patient, so midwives do not have extensive information prior to an appointment unless they have encountered the woman previously. Maternity notes are not generally linked to extra-departmental records, meaning that a history of alcohol related admissions to another service may remain unknown unless revealed by the woman.

According to NICE guidance \citep{NICE2010a,NICE2010} the issue of substance misuse should be raised at the initial booking appointment, followed by subsequent action if a concern is raised is at the discretion of the midwife. This may take the form of specific guidance to reduce intake, or if deemed necessary a referral to a specialist midwife and relevant interdisciplinary team. On alcohol consumption, policy regarding how to determine the level of consumption is at the time of writing generally at the level of local health authority, hospital trust, or according to the best judgement of the individual midwife, with no guidance provided by NICE. This commonly takes the form of average units per week, but may include \ac{T-ACE}\footnote{The \ac{T-ACE} is a four question screening test for alcohol misuse intended specifically for use with pregnant women.} \citep{Sokol1989863} and similar measures. 

Beyond the \enquote{booking} appointment, the onus is on women to raise concerns about their drinking behaviour, or the midwife to probe further if they feel it is warranted. In either case, once a concern has been raised the midwife must respond clinically, and inevitably personally, to the information.

In an ideal world, all interactions with healthcare providers would be immediately and fully disclosive, with no repercussions for the patient. However, alcohol misuse by women is known to attract stigma \citep{Gomberg1988}, and is a recognised barrier to appropriate treatment in the maternity context \citep{NICE2010,Radcliffe2011}.


\subsection{Disclosure Game}
\label{sub:the_game}

In order to translate the scenario sketched above into a more abstract, tractable form, we cast it as a signalling game, and assume that women's disclosures (or not), are signals. We also impose a discretisation on the continuum of alcohol use, and use three types of behaviour - light\footnote{Or abstinent, the extent of alcohol consumption being such that it would generally be felt to pose essentially no risk.}, moderate, or heavy. Correspondingly, they are limited in what signals they may send when claiming to be one of these three types. 

Midwives are treated in a similar fashion, where their type corresponds to how negatively they regard a drinking pattern - non-judgemental, moderately judgemental, and harshly judgemental. The expression of this judgement is not a matter of choice on their part, and is assumed to have no impact on their clinical response, which is to either refer the woman for specialist treatment, or do nothing.

At the end of a game, each player receives a payoff dependent on the actions and types of both players. Because both women and midwives have an interest in the outcome of the pregnancy, and would prefer a healthy baby, the payoff has a common interest component. Hence, both players receive a payoff based on the outcome of pregnancy, but women bear a social cost dependent on the signal they sent and the midwife's reaction to it. Similarly, midwives pay a cost if they refer to a specialist, mirroring the organisational cost of non-routine care. Table \ref{tab:Payoff-matrix} shows the three payoff matrices which together describe the game.

\begin{table}
\caption{Payoff matrices}
\label{tab:Payoff-matrix}
\subfloat[Social cost, \(X_{s}\), for women, given their signal, and the midwife's type\label{tab:Social-cost-matrix}]{%
\begin{tabular}{|c|c|c|c|c|}
\cline{3-5} 
\multicolumn{2}{c}{} & \multicolumn{3}{|c|}{Woman}\tabularnewline

\hline
\multirow{4}{*}{\rotatebox[origin=c]{90}{Midwife}} &  & Heavy & Moderate & Light\tabularnewline
\cline{2-5} 
 & Harsh & -2 & -1 & 0\tabularnewline
\cline{2-5} 
 & Moderate & -1 & 0 & 0\tabularnewline
\cline{2-5} 
 & Non & 0 & 0 & 0\tabularnewline
\hline 
\end{tabular}

}\qquad
\subfloat[Health outcome, \(X_{h}\), for women and midwives, given the midwife's action, and woman's type\label{tab:Referral-payoff-matrix}]{%
\begin{tabular}{|c|c|c|c|c|}
\cline{3-5} 
\multicolumn{2}{c}{} & \multicolumn{3}{|c|}{Woman}\tabularnewline

\hline 
\multirow{3}{*}{\rotatebox[origin=c]{90}{Midwife}} &  & Heavy & Moderate & Light\tabularnewline
\cline{2-5} 
 & Refer & 10 & 10 & 10\tabularnewline
\cline{2-5} 
 & Don't refer & -2 & -1 & 10\tabularnewline
\hline 
\end{tabular}

}

\subfloat[Referral cost, \(X_{c}\), for midwife, given their action and the woman's type\label{tab:inst_cost_matrix}]{%
\begin{tabular}{|c|c|c|c|c|}
\cline{3-5}  
\multicolumn{2}{c}{} & \multicolumn{3}{|c|}{Woman}\tabularnewline
\hline 
\multirow{3}{*}{\rotatebox[origin=c]{90}{Midwife}} &  & Heavy & Moderate & Light\tabularnewline
\cline{2-5} 
 & Refer & -9 & -9 & -9 \tabularnewline
\cline{2-5} 
 & Don't refer & 0 & 0 & 0\tabularnewline
\hline 
\end{tabular}

}
\end{table}

As an example, consider the challenge faced by an agent of the heavy drinking type. In order to get the best health outcome, they must be referred and would ideally achieve this without paying any social cost at all. The best move depends on the type, and beliefs of the midwife. For example, a particularly unlucky scenario might be for the midwife to not only be of a harshly judgemental disposition, but to believe that no women really need to be referred (i.e. that all women are light drinkers). Even a relatively weak belief in this possibility can make the honest signal look like an unwarranted risk.


%Might belong in a more methody bit? Read some stuff to check..
To formally define the game, let \(N = \{m, w\}\) be the set of players each with a private type \(\theta_{i} \in \Theta\), and a set of types \(\Theta=\{l, m, h\}\), with pure strategies \(A_{m}=\{r,n\}, A_{w}=\{l, m, h\}\). Here, \(\{l, m, h\}\) correspond to light, moderate, and heavy alcohol consumption for women, and non-judgemental, moderately judgemental, and harshly judgemental for midwives. Midwives' pure strategies \(\{r,n\}\) are to refer, or do nothing, and those for women are to signal that they have one of the possible drinking patterns.
Additionally define two utility functions - 
\begin{equation}
u_{w}(s_{w}, s_{m}, \theta_{w}, \theta_{m})=X_{s, s_{w}, \theta_{m}} + X_{h, \theta_{w}, s_{m}}
\end{equation} 
\begin{equation}
u_{m}(s_{w}, s_{m}, \theta_{w})=X_{h, \theta_{w}, s_{m}} + X_{c, \theta_{w}, s_{m}},
\end{equation} with $X_{c}$, $X_{h}$, and $X_{s}$ being the payoff matrices as in table \ref{tab:Payoff-matrix}, $s_{w}$ and $s_{m}$ denoting a specific signal by a woman, and referral response by a midwife. Lastly let \(p_{w}(l, m, h)\), \(p_{m}(l, m, h)\) be distributions over types of women, and midwives respectively.


As noted, rather than solve the game, we allow populations of agents to play it, and hence stipulate further that women are drawn in order from a queue of \(n_{w}\) women (where \(n_{w}=1000\) in all simulations), and play against a midwife chosen at random from a population of \(n_{m}\) (\(n_{m}=100\)). They play for a maximum of \(r_{w}\) rounds (\(r_{w}=12\) following the routine number of ante-natal appointments in the UK \citep{NICE2010a}) or until they are referred, and a new player is drawn from the same distribution that produced the original players to replace them. If they are not referred, they rejoin the back of the queue after their appointment. In either case, they are informed of their payoff after each round and update their beliefs accordingly using one of the rules described in section \ref{sub:the_agents}.

Midwives play for \(r_{m}\) rounds (\(r_{m}=1000\) in all experiments), and conduct appointments in parallel, i.e. if there are 5 midwives, then five women are drawn from the queue and assigned at random to the midwives. 
Unlike women, midwives are only informed of their payoff if they choose to make a referral. Both groups of agents have perfect recall, and midwives are assumed to retrospectively update their observations if they make a referral after a number of appointments.


\subsection{Social Learning}
\label{sub:info_sharing}

In reality, learning is not exclusively from personal experience, and social learning plays an important role. This social dynamic fits naturally into an agent framework, but is difficult to address without using an approach concerned with process, so we take advantage of this to show a na{\"\i}ve take on it here.

In the disclosure game model, this takes the form of having each agent recount their play history to their colleagues with some probability \(q\). Individuals then incorporate shared information into their beliefs using weighted updates, e.g. for a midwife a shared observation of a low type signal contributes to their beliefs by \(w\), and \(0\leq w\leq 1\) (i.e. \(n_{j} = n_{j} + w\)).
Women share only when they have finished play, and provide their complete history of games, because they have accurate information about the outcomes. By the same rationale, midwives share only their history with the most recent woman they referred. Sharing occurs simultaneously for all players at the end of each round, and all memories are either shared immediately or discarded.\footnote{More precisely, memories of games remain, but it is assumed that only the most current information is relevant enough to be shared.} Accounts are shared with some probability, to all fellow players. For example, a heavy drinker finishes play having claimed to be a light drinker, without ever being referred, and their account is selected to be shared with some probability $q_{w}$. 

Because of their differing problem representations, the simple payoff reasoners and their more complex counterparts incorporate this exogenous information differently. The simple payoff based rule relies on a belief structure relating actions directly to rewards which is essentially model free. Because payoffs differ by the agent's private type, the information shared may not correspond to the experience of the listening agent in the same scenario. As a result, payoff reasoners have a belief bias towards the most common player type, and can believe in outcomes that are, for them, impossible.

A payoff based agent, who is a light drinker, hears the account of the heavy drinker. They take the account as literally happening to them, and update their beliefs to include the possibility that there is a negative outcome attached to claiming to be a light drinker.

By contrast, representing the problem in terms of the probabilities of the individual lotteries imposes a model that abstracts the new information from payoffs, and allows the agent to discard implausible outcomes. This stronger assumption as to the static and known qualities of payoffs does however reduce the flexibility of the decision rule.

Returning to our example, a light drinker using this decision rule would follow the account through from their position in the game tree, correctly inferring that the outcome in their case would be positive.