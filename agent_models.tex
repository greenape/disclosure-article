\subsection{Agent Models}
\label{sub:the_agents}
%Outline basic structure, then specifics on each one.
%This should probably go lexic -> bayes payoff -> bayes -> CPT in order
% of complexity.
%suggest there are added shells of complexity
%		mention the info sharing because homophilly
%			related - the enhancement to the bayes/cpt agents is that they personalise, as much as anything.
%		properly explain dirichlet priors
% How about comparing the decision problem representation for the agent types?

While in principle a wide variety of agent models are possible, given that decision rules operate on essentially the same information, and produce the same output - a decision, we limit ourselves here to four. The simplest is a lexicographic rule (1), in the spirit of a \ac{FFH} \citep{Gigerenzer2004} which uses only information about payoffs given actions; this is followed by a Bayesian risk minimisation rule (2) using the same information; a second Bayesian risk rule (3) which uses information about the underlying lottery; and a two-stage \ac{CPT} \citep{Hau2008} agent (4) which is identical to 3, but uses the \ac{CPT} decision rule from \cite{Tversky1992}. Hence, each successive decision model adds a layer of sophistication to the problem representation while retaining the same input-output characteristics.

Agents have perfect recall, midwives recognise women if they repeatedly encounter them, making use of new information for retrospective updates. However, all four agent models make decisions `as-if' they were always facing a new \enquote{opponent}.

A simplifying assumption is made that all midwives have just qualified after receiving identical training. As a result, they have homogeneous beliefs about  women and assume to some extent that they are honest.
Women have heterogeneous beliefs, which correspond to experiencing \(k\) randomly chosen paths through the game, and following each path at least once.


\subsubsection{Lexicographic Heuristic}
\label{sub:lexico}

The lexicographic heuristic \citep{Luce1956,Fishburn1974} (algorithm \ref{alg:lexico}) follows the form of that used in \cite{Hau2008}, and assumes a simplified problem representation, where an action is a choice between combined lotteries. Functionally, the heuristic maintains a count of the number of times that each action was followed by a payoff, and chooses the action which most commonly has the best payoff, i.e. one reason decision making. Where there is no clear best action, but one or more is evidently worse, a choice must be made as to whether to discard the poorer action; in this case we have elected to retain it.
This approach requires minimal computation, and does not assume that \(u_{i}\) is static, or known.

Women resolve this by approximating the utility function, as a function \(f(s_{w}, \sigma)\) on their choice of signal and an unknown distribution $\sigma$, which maps to \(u_{w}\) -- i.e. \(s_{w}\) is a choice between simple lotteries. The algorithm maintains a count, \(n\), of the number of occurrences of each outcome given the choice from \(s_{w}\).

Midwives solve a slightly different problem with more information, where \(s_{w}\) is known, and \(s_{m}\) is the lottery choice -- \(f(s_{w}, s_{m},\sigma)\). This is resolved by maintaining a separate count for each signal (i.e. \(n_{s_{w},s_{m}}\)), and otherwise following the same algorithm (\ref{alg:lexico}).

\begin{algorithm}
\begin{algorithmic}
\State $n \gets 1, action \gets none$
\While{$action = none$}
\State Calculate the $n$th most common outcome following each action.
\State Sort actions by the value of the $n$th most common outcome.
\If{clear winner} \State $action \gets best$ \EndIf
\State $n \gets n + 1$
\EndWhile
\State \Return action
\end{algorithmic}
\caption{Psuedocode for the Lexicographic heuristic\label{alg:lexico}}
\end{algorithm}

\subsubsection{Bayesian Payoff}

The Bayesian payoff agent uses the same subset of information as the lexicographic method, but updates beliefs on the link between actions and payoffs using the Bayes rule, and attempts to choose the action which minimises risk.

Given the discrete nature of actions and payoffs, coupled with a desire for tractability of the
simulation, the Dirichlet distribution is employed as a prior to represent these beliefs. 
The distribution is particularly convenient, in that to infer the
probability of a signal implying a payoff is
simply -

\begin{equation}
p(x=j|D,\alpha)=\frac{\alpha_{j}+n_{j}}{\sum_{j}(\alpha_{j}+n_{j})}\label{eq:posterior}
\end{equation}


Where \(n_{j}\) is simply the count of occurrences of signal-payoff pair \(j\), and \(\alpha_{i}\) is the pseudo-count of prior observations\footnote{Psuedo-counts are related to, but distinct from prior beliefs. Here, the pseudo-count is a parameter of the prior Dirichlet distribution and is nothing more than a hypothetical count of prior observations. The outcome is a result of the relationship between $\alpha$, and $n$, which is increasingly driven by the priors as fir higher values of $alpha$.} for a pair \(i\). Hence,
the belief that a signal will lead to a payoff is the number
of times that pairing has been observed (including the pseudo-count),
over the total number of observations thus far. This makes computation
of beliefs fast and simple, since all that must be maintained is
a count of observations.
As before, midwives follow a similar pattern but maintain \(n_{s_{w}}\) independent counts of pairings between referral choice and payoff, updating their beliefs about the relationship between the choice to refer and payoff given the signal they have received.

Agents then choose the strategy $s_{i}$ to minimise risk $R_{i}$, which is simply defined as - 
\begin{equation}
R_{w}(s_{w}) = \sum_{x \in X} -xp(x | s_{w})
\label{eq:women_asker_risk}
\end{equation}
\begin{equation}
R_{m}(s_{w}, s_{m}) = \sum_{x \in X} -xp(x | s_{w}\wedge s_{m}),
\label{eq:mw_decider_risk}
\end{equation}

where $X$ is the set of payoffs, $x$, the agent has observed to follow $s_{i}$.

\subsubsection{Bayesian Risk Minimisation}

The second Bayesian agent augments the reasoning of the simple payoff model, making the stronger assumption that the utility function is static, and known. Women maintain two sets of beliefs, corresponding respectively to \(p_{m}\), and the probability of referral given signal choice. This leads to the risk function $R_{w}$, minimised with respect to \(s_{w}\) -

\begin{equation}
R_{w}(s_{w}, \theta_{w}) = \sum_{i\in A_{m}}\sum_{j\in \Theta} -u_{w}(s_{w}, i, \theta_{w}, j)p(j)p(i | s_{w}),
\end{equation}

so that the risk of a signal is the sum of the products of all payoffs with the probabilities of their entailed midwife types and responses.

Midwives reasoning centres on determining the meaning of signals, since given the knowledge of what some signal \(s_{w}\) conveys about the true type of the sender, the payoff for an action is known. As such, their inference process is the same as for the simple Bayesian agent but over signal-type pairs, and they attempt to minimise the following risk function $R_{m}$, minimised with respect to \(s_{m}\) -

\begin{equation}
R_{m}(s_{w}, s_{m}) = \sum_{i\in \Theta} -u_{m}(s_{w}, s_{m}, i)p(i | s_{w})
\end{equation}

\subsubsection{Descriptive Decision Theory}

The most complex decision rule used is \ac{CPT}, which attempts to reproduce a number of systematic deviations from rationality observed in humans. Rather than risk, `prospects', the sequence of payoff-probability pairings in ascending order of payoff, associated with an action  are used as decision criteria. While \ac{CPT} has primarily been applied in the context of decisions from description, it has been modified to deal with decisions from experience by incorporating a first stage where probabilities are estimates from observations \cite{FoxCPT}. In this instance the Bayesian inference process fills the first stage role.

%Maybe cut this bit and make explanation richer?
\ac{CPT} uses transformed probabilities, under weighting small probabilities, and over weighting large ones. This is intended to reflect the observed behaviour of humans, where sufficiently high likelihoods are treated as certain, and in contrast, low probabilities as impossible. The correct weighting function is subject to some debate, but here we have used that of \citet{Tversky1992}, which treats probabilities differently for gains (\ref{eqn:cpt_p_pos}) and losses (\ref{eqn:cpt_p_neg}):

\begin{align}
w^{+}(p) & = \frac{p^{\gamma}}{(p^{\gamma}+(1-p)^{\gamma})^{\frac{1}{\gamma}}}\label{eqn:cpt_p_pos}\\
w^{-}(p) & = \frac{p^{\delta}}{(p^{\delta}+(1-p)^{\delta})^{\frac{1}{\delta}}}\label{eqn:cpt_p_neg},
\end{align}


where $p$ is the unweighted probability, and $\gamma$ and $\delta$
are the weights for gain and loss probabilities respectively. Along similar lines, the values of losses and gains are transformed to reflect a tendency to regard a loss as more significant than a gain, such that -

\begin{equation}
v(u_{i})=\begin{cases}
f(u_{i}),& \text{if}\, u_{i}>0\\
0,& \text{if}\, u_{i}=0\\
\lambda g(u_{i}),& \text{if}\, u_{i}<0
\end{cases},
\end{equation}


where

\begin{equation}
f(u_{i})=\begin{cases}
u_{i}^{\alpha},& \text{if}\,\alpha>0\\
\ln(u_{i}),& \text{if}\,\alpha=0\\
1-(1+u_{i})^{\alpha},& \text{if}\,\alpha<0
\end{cases}
\end{equation}
\begin{equation}
g(u_{i})=\begin{cases}
-(-u_{i})^{\beta},& \text{if}\,\beta>0\\
-\ln(-u_{i}),& \text{if}\,\beta=0\\
(1-u_{i})^{\beta}-1,& \text{if}\,\beta<0
\end{cases},
\end{equation}


and $\alpha$ and $\beta$ are respectively the power of a gain
and a loss, and \(\lambda\) is a multiplier giving the aversion to loss.

Finally, the transformed probabilities are used to construct decision weights, \(\pi^{+},\pi^{-}\) for losses and gains, where

\begin{equation}
\pi_{n}^{+}=w^{+}(p_{n})
\end{equation}
\begin{equation}
\pi_{-m}^{-}=w^{-}(p_{-m})
\end{equation}
\begin{equation}
\pi_{i}^{+}=w^{+}(p_{i}+\ldots+p_{n}) - w^{+}(p_{i+1}+\ldots+p_{n}),0\leq i \leq n - 1
\end{equation}
\begin{equation}
\pi_{i}^{-}=w^{-}(p_{-m}+\ldots+p_{i}) - w^{-}(p_{-m}+\ldots+p_{i-1}),1-m\leq i \leq 0
\end{equation}

The \ac{CPT} value of a single outcome prospect \(f=(u_{i};p_{i})\), is $v(u_{i})\pi^{+}(p_{i})$
if $u_{i}\geq0$, and $v(u_{i})\pi^{-}(p_{i})$ otherwise. For any given action the \ac{CPT}
value, \(V\) is the sum of the value of the prospects of that action, as
in the Bayesian risk model, and the agent chooses the option which maximises this quantity.